%!TEX root=../../paper.tex

\begin{table}[p]
\large
\centering
\captionsetup{width=.9\columnwidth}
\caption{Microarchitecture configurations for P1 and P2 in
\Fig{fig:ivso}. They represent different energy-delay trade-offs. For comparison purpose, we also show the parameters for ARM Cortex-A15, whose information is gather from measurements using the 7-Zip LZMA Benchmark~\cite{7cpu-a15} and ARM's public presentation~\cite{a15-slide}.}
\renewcommand*{\arraystretch}{1.4}
\renewcommand*{\tabcolsep}{15pt}
\resizebox{.9\columnwidth}{!}
{
	\begin{tabular}{l c c c}
	\toprule[0.15em]
        ~      & \bigstrut\textbf{P1} & \bigstrut\textbf{P2} & \bigstrut\textbf{Cortex-A15}\\
	\midrule[0.05em]
        Issue width						&	1		&	3	&	3	\\
        \# Functional units				&	2		&	3	&	8	\\
        Load queue size (\# entries)		&	4		&	16	&	16	\\
        Store queue size (\# entries)	&	4		&	16	&	16	\\
        BTB size (\# entries)   &       1024    &       128 &   64       \\
        ROB size (\# entries)			&	128		&	128	&	40+	\\
        \# Physical registers			&	128		&	128	&	?	\\
        L1 I-cache size (KB)				&	64		&	128	&	32	\\
        L1 I-cache delay (cycles)		&	1		&	2	&	?	\\
        L1 D-cache size (KB)				&	8		&	64	&	32	\\
        L1 D-cache delay (cycles)		&	1		&	1	&	4	\\
        L2 cache size (KB)				&	256&	1024	&	512\textasciitilde4096	\\
        L2 cache delay (cycles)			&	16		&	16	&	21	\\
	\bottomrule[0.15em]
    \end{tabular}
}
\label{tab:dse:ednp}
\end{table}
