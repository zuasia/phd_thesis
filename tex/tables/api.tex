% !TEX root = ../../paper.tex

\begin{table}[p]
\large
\centering
\caption{Specifications of the \greenweb APIs. Each API is a new CSS rule specifying the QoS information when a particular event is triggered on certain Web application element.}
\renewcommand*{\arraystretch}{1.1}
\renewcommand*{\tabcolsep}{8pt}
\resizebox{\columnwidth}{!}
{
    \begin{tabular}{l l}
    \toprule[0.15em]
    \bigstrut\textbf{Syntax} & \bigstrut\textbf{Semantics} \\
    \midrule[0.05em]
    \specialcell{\texttt{\textcolor{blue}{E:QoS} \{}
\\~~~~\texttt{\textcolor{Green}{onevent-qos:} continuous}
\\\texttt{\}}} & \specialcell{As soon as \texttt{onevent} is triggered on\\DOM element \texttt{E}, the application must\\continuously optimize for frame latency.\\Use the $P_I$ and $P_U$ values in \Tbl{tab:qos_info} as\\the default QoS target for all frames.}\\
    \midrule[0.05em]
    \specialcell{\texttt{\textcolor{blue}{E:QoS} \{}
\\~~~~\texttt{\textcolor{Green}{onevent-qos:} single,}
\\~~~~\texttt{~~~~~~~~~~~~~~short|}
\\~~~~\texttt{~~~~~~~~~~~~~~~long}
\\\texttt{\}}} & \specialcell{Once \texttt{onevent} is triggered on element\\\texttt{E}, the application must optimize for the\\latency of the single frame caused by\\ \texttt{onevent}. Users expext short (long)\\latency. Use the $P_I$ and $P_U$ values in\\\Tbl{tab:qos_info} as the default QoS target.}\\
    \midrule[0.05em]
    \specialcell{\texttt{\textcolor{blue}{E:QoS} \{}
\\~~~~\texttt{\textcolor{Green}{onevent-qos:} continuous|}
\\~~~~\texttt{~~~~~~~~~~~~~~~~~single,}
\\~~~~\texttt{~~~~~~~~~~~~~~~ti-value,}
\\~~~~\texttt{~~~~~~~~~~~~~~~tu-value}

\\\texttt{\}}} & \specialcell{Explicitly specify $P_I$ (\texttt{ti-value}) and\\$P_U$ values (\texttt{tu-value}) for QoS targets.\\Note that both values must either appear\\or be ommitted together.}\\
    \bottomrule[0.15em]
    \end{tabular}
}
\label{tab:api_spec}
\end{table}
