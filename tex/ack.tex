%!TEX root=../paper.tex

\begin{acknowledgments}		% Optional
%\index{Acknowledgments@\emph{Acknowledgments}}%

As I march towards the end of my doctoral study, I feel more connected to the intriguing campus of The University of Texas at Austin. The past five years at UT is life-changing. Five years in the twenties may be the most precious period of time in a man's life, and I really appreciate the happiness as well as the pains of exploring frontier computer architecture knowledge during this period of time in the lovely city of Austin, TX.

In all cultures around the globe, it is an honorable pursuit for a person to choose truth and knowledge as his/her lifelong career. As a humble young man who just graduated from college, I also had the unsullied dream of being a scholar who will be of use the mankind when I embarked on my Ph.D. journey five years ago. Looking back at this point, I can proudly say it was a truly a tremendously valuable investment that I dedicated myself into UT's rigorous Ph.D. program. Texas is a young place, where the spirit of cowboys striving to survive the wild and to fight for justice is still retained. I am lucky enough to receive an honest and rigorous academic training at the University of Texas, like the immaculate virtue of a newborn child. I hope in the rest of my life I will retain some of the qualities I learned during my Ph.D. study at UT Austin, including the ambition of purse truth and justness, and the habit of being conscientious and honest.  

%Thank vj, for his help
I wish to thank the multitudes of people who have helped me on my Ph.D. journey. First and foremost, I want to give my sincere gratitude to my advisor, professor Vijay Janapa Reddi, who guided me to be a professional computer architect from an immature fresh college graduate. Five years ago, I came to UT as a student primarily trained to develop integrated circuit at Shanghai Jiao Tong University, with the hope of finding an advisor who would help me with my Ph.D. funding, and who would help me find a promising research topic that can lift my system layer up to architecture design, from the low level circuit implementation. Vijay offered me a precious opportunity the first week I came to Austin. We worked very hard together during my first year at UT, which set a good foundation for my upcoming Ph.D. experience. Vijay is a very responsible mentor, he works hard to make sure all student's Ph.D. research are properly funded, so that we students can focus on our work whole-heartedly. Vijay also help me, as well as other students, find various internship opportunities so that we can witness how the real world works and learn to transition into the professional life style from a student role. I deeply appreciate all the financial and career help Vijay provided me with.

The journey of completing a Ph.D. is not easy. As my mentor, Vijay patiently taught me how to choose research areas that are of value in the future, and how to produce high-quality research papers. His vision, and determination to achieve things inspires me a lot. Along my Ph.D. journey, I learned that scientific research takes a lot of perseverance. It involves constantly overthrowing our own conjectures, trying out different ideas, and eventually making things work. In this process, Vijay would always find me the correct people to connect with. I still remember in my first independent research project Vijay repeatedly drove me to IBM research to discuss with Charles when we encountered obstacles. It is in those experiences when I realized that intelligent discussions are key to innovation, because it is when one combines knowledge from different areas that inspiration occurs. This idea gradually shaped my mindset beyond just becoming a better technical developer, but to becoming a more active communicator, which I will forever benefit from.

%Thank Charles, for his guidance
I want to give special thanks to Charles R. Lefurgy, who played a critical role in my Ph.D. study. My research topic and methodology involves comprehensive hardware measurement and instrumentation, which is very challenging for fresh graduates who have no experience with complex production microprocessors. In my first research projects, Charles patiently guided me on how to get the correct measurement done, how to interpret the data, and where the problem is. I still remember those long phone discussion with Charles in my second year at UT. I remember that when we were submitting my first paper to MICRO in 2015, Charles sat with me late in the night in my cube at the P.O.B building, and we both hadn't had dinner. His humor helped me relieve the pressure of editing the paper, and he patiently helped me pick up the wording and grammar mistakes in my draft. Without Charles's help, we couldn't imagine how I could have initiated my own research, no to mention how I could have progressed towards the finish line. Charles is a wonderful, nice senior to work with. He loves his research fruition and likes mentoring young students even though he is very busy with his own work at IBM research. Charles sets a good example for me on how to be a nice college in my future career. I feel very fortunate to have the pleasure of working with him in my Ph.D. study. 

%Thank the other committee members, and other faculties at UT
I would also like to thank all other members of my dissertation committee: Mattan Erez, Lizy John, and Andreas Gerstlauer. Their contentiousness, keen insights, and feedbacks are absolutely invaluable in completing my Ph.D. thesis. In addition, Lizy taught me Computer Performance Evaluation and Benchmarking, and Mattan taught me Computer Architecture Parallelism and Locality; both helped me form a solid foundational computer architecture knowledge. Mattan is a smart, positive, and easy-going mentor who I admires. Andreas makes me realize never feel self-contained and always do things thoroughly when he helped me refine my thesis. I truly appreciate the rigor Andreas possesses. Lizy is always nice and helpful, special thanks to her for serving on my committee. The courses these committee members provide, and the way they mentor their own students make me realize how a top-notch graduate school program functions at UT.

During my time at UT, I also had the privilege to interact with other faculty members, in particular Yale Patt. I had the fortune of following Yale in his micro-architecture course and witnessing how a highly respected scholar emphasizes on teaching, on nurturing seed-like ideas in research, and on keeping an eye on every detail in all the work. 

%Thank Wei, Alper, Bin, for their internship help
I had four internships during my Ph.D. study, and I want to thank all my mentors during each internship. Sek Chai was my mentor at SRI International. I thank him for opening my eyes to the industry for the first time. Wei Huang and Indrani Paul were my mentors during my part-time internshis at AMD Research. I really appreciate the help and convenience they provided me with during my research project here. We together won the IEEE MICRO top picks award for our joint work there, and the work I did with them served as a critical milestone in my Ph.D. journey. Alper Buyuktosunoglu was my mentor during my internship at IBM Thomas.J.Watson research center. He and his colleague's experience and sharp insights on computer architecture truly opened my eyes. Bin Li was my mentor when I interned at Facebook, I thank him for his warmheartedness when providing job seeking advice for me as I approach my graduation point.


%Thank Jingwen, and other friends for their help
My friends and lab mates at UT Austin are a great support of mine that let me go through all the setback and challenges in research, and in life. Words cannot describe my gratitude to them. I would like to give especial thanks Jingwen Leng, who was like a big brother to me when I first started working with Vijay. Jingwen helped me understand what a Ph.D. program is like at UT ECE and what it means to be a Ph.D. student before I came to Austin. He kept encouraging me when I fell into setbacks and gave me a lot of hands-on advice. I thank him for introducing me to Vijay, without which I couldn't have gone through this invaluable experience. Jingwen deeply influenced me on how to be a helpful and responsible senior student for upcoming new students, and I always look up to him. I enjoyed a lot of intelligent discussion with Yuhao, not only on computer science research, but also on many other areas of life. Yuhao is a clever guy with a lot of curiosity, best of wishes to him on his academic career. I had a good time with Wenzhi Cui and Daniel Richins in the later stage of my study. Wenzhi brought new angles of looking at problems to me, and Daniel helped me edit my last research paper patiently.

I want to express my deepest appreciation to my friends Zhengcheng Tao, Yuanqi Chen, Xiaojun Lin, Jichao Chen, Chuang Wang, as well as those anonymous friends who gave me hope and encouragement over phone when I was at the bottom. You are all very nice people with warm hearts, and we've had a lot of fun together in our spare time. Wish you all the best wherever you go.

%Last but not least, thank my mother
Most importantly, not a single word of this dissertation would happen without the nurture of my loving mother, Zhijie Li. Ph.D. is not only an academic experience, but also a life journey where one experience failures and life difficulties. In this process I gradually grew up, and realized the difficulty of raising up a child. In addition, my mother's perseverance of continuing pursuing her literature dream after retirement has truly inspired me. I cannot express how amazed I am when I read the poems she wrote in her spare time. I appreciate my mother not only as a son, but as a researcher and scholar who values knowledge, beauty, and truth. It is easy for people to enjoy the pleasure of life without the pressure of work, yet my mother choose to read, absorb, and write, simply chasing her youth dream. My mother is a lovely and wonderful human being, and I am very proud of her. May she continue to have a happy life.


\end{acknowledgments}
