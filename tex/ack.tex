%!TEX root=../paper.tex

\begin{acknowledgments}		% Optional
%\index{Acknowledgments@\emph{Acknowledgments}}%

As I march towards the end of my doctoral study at The University of Texas at Austin, I'm feeling more connected to this simple, clean, yet intriguing campus. The past five-year experience of exploring frontier computer engineering knowledge at UT is life-changing. Five years in a person's twenties are one of the most precious period of time, and I feel it is a tremendously valuable investment that I dedicated myself into the rigorous Ph.D. program here at UT ECE.

The journey of completing a Ph.D. is pivotal. Along my Ph.D. path, I learnt that scientific research takes tremendous perseverance because it constantly involves overthrowing conjectures, trying out various ideas and proposals, and eventually acquiring the truth. I also learnt that scientific research needs great wisdom, because it is when one combines knowledge from different areas, and switches the angles of looking at a problem that inspiration and criticality occurs. Having gone through this process undoubtedly affects how I view the world, and how I make major life decisions. 

It has been my great fortune to conduct my doctoral study on computer architecture at The University of Texas at Austin. 

% different faculties, world-class engineers and scienctists taught me the scrutiny and visionary of doing research

% in particular, my advisor vj reddi

% 



I wish to thank the multitudes of people who helped me throughout this journey. First and foremost, I would like to thank The University of Texas at Austin, and the taxpayers of the great state of Texas, for supporting me to pursue a degree in a filed that I am truly passionate about. Among many things, UT Austin provides a stimulating, diverse, and welcoming environment for intellectual explorations that one could ever imagine having.

I would like to express my sincere gratitude to my advisor Vijay Janapa Reddi, who turned me from a student to a researcher. I came to UT as an avocational computer architecture enthusiast. Vijay patiently taught me how to turn half-baked ideas to high-quality research papers, and how to turn individual papers into a dissertation. I appreciate the many technical and visionary discussions that he had with me while walking on campus.

Vijay also trained me to be a long-term investor. In my first semester working with Vijay, he spent over a thousand dollars to send us to a half-day course on information visualization. He later sent us to other venues such as training mental agility. Although some of these investments might have not paid off immediately toward a paper---some never will---it gradually shaped my mindset beyond just becoming a better graduate student to becoming a better person, which I will forever benefit from.

I would also like to thank other members of my dissertation committee: Lizy John, Derek Chiou, Christine Julien, and Scott Mahlke. Their keen insights and feedbacks are absolutely invaluable in developing and refining the concepts in my dissertation. In addition, Lizy taught me Computer Performance Evaluation and Benchmarking, and Derek taught me Parallel Architecture; both helped me form the computer architecture foundations. Special thanks to Christine and Scott who gave me many pieces of advice and wrote me recommendation letters for my job search.

During my time at UT, I also had the privilege to interact and work with other faculty members, in particular Mattan Erez, Mohit Tiwari, and Yale Patt. Mattan is a calm, sharp, and encouraging human being who I look up to. He taught me Computer Architecture during my first semester in UT (and this country), which is the reason that I now can talk comfortably about computer architecture. Mohit is always there to provide me lots of advice in career development. I worked with Dr. Patt during my first year as a graduate researcher and as a teaching assistant for his Computer Architecture course in my second semester. The interactions with him provided me unique perspectives that I would have gotten from nowhere otherwise.

I had four summer internships during my graduate study, and I want to thank my mentors during each internship. Osvaldo Colavin was my mentor at STMicroelectronics in summer 2011. We first met at DAC that year at San Diego, and he later made me an internship offer. I thank Osvaldo for truly opening my eyes to the industry for the first time. Mauricio Breternitz was my mentor during my two internships at AMD Research in summer 2012 and 2013. I admire his incredible experience in hardware and software industry. He gave me the opportunity to integrate several AMD specific hardware performance counters into the \texttt{libpfm} in my first internship, and gave me lots of freedom to explore Web browser caching during my second internship. Nat Duca was my mentor during my internship at the Google Chrome team in summer 2015. Even to this day I couldn't believe that I had worked with Nat. Nat is an incredible person and great friend who continues to help me after the internship. He also connected me with many other wonderful Chrome engineers---the reason that the Web is well alive and kicking!

My lab mates and friends at UT Austin not only help me academically, but also always encourage me to never back off from challenges, and I greatly thank them for that. Jingwen Leng and Aditya Srikanth helped me with my first project to get my research started. Jingwen and I had a lot of fun beyond research. Among many things, we often drove to Houston just to play snooker, and flew to UK to watch snooker games. Daniel Richins and Wenzhi Cui helped me with the Node.js project. Wenzhi would always listen to my nonsense ideas. Matthew Halpern helped me with \textit{everything} since he joined. He acts as a cheerleader when I am down and keeps me sane. I have also met many friends at UT and Austin who I want to thank: Song Zhang, Haishan Zhu, Tianhao Zheng, Yazhou Zu, Yong Li, Minsoo Rhu, Milad Hashemi, Faruk Guvenilir, and Khubaib. Special thanks to Khubaib who mentored me during my first year working with Dr. Patt and continued to support me after that.

I am grateful to the Google Ph.D. fellowship for supporting my Ph.D. research. Matt Welsh is my fellowship mentor, who is a true enemy of hogwash. Matt has supported my research throughout my entire graduate study by being an advocator inside Google for our research. He also generously wrote a recommendation letter for my job search. I was also supported by the Microelectronics and Computer Development Fellowship at the Cockrell School of Engineering in UT Austin during my second year.

I also want to thank my undergraduate advisor Yangdong (Steve) Deng at Tsinghua University. I was an undergrad at Beihang University, and Steve just came back from the United States to start his lab. I anxiously sent him a cold email expressing my interests, and (not) surprisingly he decided to bring me to the lab. We together worked on GPGPU programming and microarchitecture, which greatly raised my interests in system architectures. He encourages me to think big and makes me believe that I, too, could publish first-class research results. I would never get a Ph.D. if not for the one year working and interacting with Steve.

Most importantly, not a single word of this dissertation would happen without the support of my loving family. My parents, Qinxiong Zhu and Xiaojing Yu, care about me more than themselves and never hold me off in pursuing whatever I am passionate about, even if that means they only saw me four times in the past seven years. Finally, I thank my wife Yunjing Luo for always being there since we first met in 2012. Although she believes that I could have finished Ph.D. sooner without her, I believe otherwise.

% world view of moderin technology, from the view of China, a country with long ancient histroy, meanshile going thourgh fastly trasitioning

\end{acknowledgments}
