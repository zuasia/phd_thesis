%!TEX root=../paper.tex

\chapter{Retrospective and Prospective Remarks}
\label{sec:conc}

This chapter provides retrospective and prospective views of my dissertation work. The retrospective part (\Sect{sec:conc:retro}) distills three principles developed from my work on building a high-performance while energy-efficient mobile Web computing system. The prospective part (\Sect{sec:conc:pros}) suggests next steps for generalizing the principles and outlines lucrative research items.

\section{Retrospective}
\label{sec:conc:retro}

As a promising first step, my proposal explores the feasibility of a holistic system design to improve the energy-efficiency of mobile Web computing while delivering user satisfaction. It argues that the traditional interfaces across the Web computing stack should be enhanced with new abstractions for QoS and hardware. Overall, it demonstrates three general principles:

\begin{itemize}
  \item \textsc{Empowering Web Developers Without Overburdening Them} Web developers today must be conscious of energy-efficiency because of increasing user awareness. Current application/runtime abstractions, however, do not provide developers opportunities to optimize for energy-efficiency. Pure runtime-based techniques are QoS-agnostic. A key principle in my work is that developers should be integrated into the energy optimization loop. \greenweb (\Sect{sec:lang}) demonstrates a first step by empowering developers to express user QoS expectations as program annotations. The QoS annotations guide the runtime to make a calculated trade-off between performance and energy consumption. Meanwhile, to incentivize a wide usage of \greenweb-like annotations, it is critical to ease developers' manual annotation efforts. \autogreen framework explores the feasibility of automatic annotation with promising results.
  
  \item \textsc{Exposing Architecture Details Without Losing Generality} Mobile system-on-chips (SoC) undergo rapid design iterations with each generation incorporating more complex cores and IP blocks. A Web runtime system must embrace, but does not overly couple with, the unprecedented hardware complexity. A guiding principle for system designers is that the runtime/architecture interface should be enhanced with new abstractions---abstractions that expose enough hardware details while not imposing too strict constraints on runtime implementation. \webrt (\Sect{sec:runtime}) is a concrete example of this principle where processor core type and core frequency in an ACMP architecture are exposed as two new abstractions to the Web runtime. The two new hardware abstractions enable a large performance-energy scheduling space while do not impose any constraints how the scheduling should be implemented.
  
  \item \textsc{Balancing Programmability With Domain Specialization} Ultimately mobile processor architecture designs will have to improve both the performance and energy-efficiency simultaneously, not just making trade-offs between the two design goals. Architectural specialization comes as the first choice to achieve both improvements. However, I argue that retaining the general-purpose programmability during the specialization process is of critical importance for the Web stack because of its inherent complexity. \webcore (\Sect{sec:arch}) demonstrates one approach of balancing programmability and specialization by starting from a (well-customized) general-purpose design and incrementally incorporating modest specializations for the most lucrative software targets.
\end{itemize}

\section{Prospective}
\label{sec:conc:pros}

The future of mobile Web computing is undoubtedly exciting. It is exciting not only because of the intellectual challenges it poses, but also because of its applicability to our everyday life and its profound impact on our society. Guided by the principles described in the previous chapter, I outline a few open problems that are critical to the next era of mobile Web computing.

\begin{itemize}
  \item \textsc{Composability of QoS Abstractions~~} As mobile application domains such as virtual reality and augment reality become more mainstream, new forms of user interaction (e.g., through nose~\cite{nosewatch}, visible light~\cite{license}) will emerge and as such users will have new ways of assessing their QoS experience. Although the QoS type and QoS target abstractions proposed in this work are sufficient for expressing QoS specifications in \textit{today}'s mobile applications, it is not clear how to express a new QoS abstraction in the context of \textit{future} mobile applications.
  
  To express semantics of new QoS specifications, the next step of Web language research should understand how to design ``primitive'' QoS abstractions, based on which complex, higher-level QoS abstractions can be easily composed. The composability of QoS abstractions is critical because enumerating every single possible kind of QoS experience in a Web programming model is not scalable. Instead, the Web programming model should ideally provide a QoS primitive library that lets developers construct different QoS specifications in a completely customized way.
  
  To achieve this goal will likely involve an intimate collaboration between the system and human-computer interaction community. Breakthroughs in neuropsychology research that seeks to construct fundamental human perception models will also play an important role.
     
  \item \textsc{Scalable Web Runtime~~} If today's hardware systems that a mobile Web runtime has to manage, e.g., an ACMP architecture, are already complex, the complexity in tomorrow's mobile hardware will be unprecedented. ITRS projects that the number of specialized IP blocks will reach upwards of 100$X$ more than today by 2022. What further adds to the hardware complexity is the fragmentation issue where, for instance, one IoT device will have non-standard, vendor-customized, and application-specific hardware components that are not found in any other devices.
  %For instance, Apple's mobile SoCs have dozens of specialized IP blocks occupying more than half of the die area~\cite{cachehw}.
  
  Future mobile Web runtime should effectively manage the hardware complexity in order to fully take advantage of the hardware capability. The challenge of the runtime design is one of scalability. On one hand, a monolithic design that appeals to all existing IP blocks is unscalable because the performance and energy overheads accumulate as the number of IP blocks surges. On the other hand, a completely customized runtime tailored for the specifics of a particular device is unscalable either. This is because software vendors would have to distribute different runtimes based on different device capabilities, essentially transferring the burden of managing hardware complexity to managing software complexity.
  
  An ideal mobile Web runtime should strike a balance between the two extremes, allowing one piece of software to be distributed across all devices while providing flexibilities to support hardware components unique to each device. One promising approach is to borrow the principles of the microkernel-based OS design: a minimal runtime kernel equipped with extensible modules, each dealing with one or one group of hardware IPs. The extensibility, i.e., the ability to (un)load a runtime module with isolated concerns, is what makes this runtime design scalable. The research challenge is to carefully select what tasks go into the runtime kernel and to define what the abstractions at the kernel-module interface should be.
  
  \item \textsc{Programmable Accelerators for Mobile Machine Learning} Machine learning techniques are making a foray into mobile/edge computing, and will be an indispensable component in future mobile Web applications with the help of extensive library support (e.g., ConvNetJS~\cite{convnetjs}). Most research in the architecture community focuses on accelerating a rather narrow set of machine learning techniques (e.g., convolutional neural networks) and application domains (e.g., image processing). There is wider space of algorithms (e.g., unsupervised learning, recurrent neural networks) and application domains (e.g., speech, language) for which computational characteristics are yet to be well-understood and hardware solutions are yet to be found.
    
  There is a need to design architectures that offer better performance and energy-efficiency on a broad set of machine learning tasks. The design should be guided by the same principle of \webcore, i.e., to balance general-purpose programmability with specialization. One particularly promising approach is to start by designing accelerator building blocks, which are then composed together to form a high-level architecture. The building blocks should exploit the fundamental computational characteristics and communication patterns common to various machine learning techniques. The composition process should allow flexible reconfiguration of different building blocks without losing much efficiency.
  
  The idea is closely related to prior explorations of configurable processors (e.g. CCA~\cite{cca}, CGRA~\cite{cgra}, LSSD~\cite{lssd}). However, the unique characteristics of machine learning tasks, especially the extremely high memory pressure, will most certainly yield new design insights and trade-offs.

\end{itemize}















