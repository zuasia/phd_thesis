%!TEX root=../paper.tex

% The abstract is required. Note the use of ``utabstract'' instead of
% ``abstract''! This was necessary to fix a page numbering problem.
% The abstract heading is generated automatically.
% Do NOT use \begin{abstract} ... \end{abstract}.
%
\utabstract
\indent

Improving power/performance efficiency is critical for today's microprocessors. From edge devices to datacenters, lower power or higher performance always produces better systems, measured by lower cost of ownership or longer battery time. This thesis studies improving microprocessor power/performance efficiency by optimizing the pipeline timing margin. In particular, this thesis focuses on improving the efficacy of \textit{Active Timing Margin}, a young technology that dynamically adjusts the margin.

Active timing margin trims down the pipeline timing margin with a control loop that adjusts voltage and frequency based on real-time chip environment monitoring. The key insight of this thesis is that in order to maximize active timing margin's efficiency enhancement benefits, synergistic management from processor architecture design and system software scheduling are needed. To that end, this thesis covers the major consumers of pipeline timing margin, including temperature, voltage, and process variation. For temperature variation, the thesis proposes a table-lookup based active timing margin mechanism, and an associated temperature management scheme to minimize power consumption. For voltage variation, the thesis characterizes the limiting factors of adaptive clocking's power saving and proposes application scheduling to maximize total system power reduction. For process variation, the thesis proposes core-level adaptive clocking reconfiguration to automatically expose inter-core variation and discusses workload scheduling and throttling management to control critical application performance. 

The author believes the optimization presented in this thesis can potentially benefit a variety of processor architectures as the conclusions are based on the solid measurement on state-of-the-art processors, and the research objective, active timing margin, already has wide applicability in the latest microprocessors by the time this thesis is written.


%The problem we address is that the timing margin in conventional microprocessors is a static value determined during chip design stage, and is significantly over-provisioned to protect against hypothetical ``worst-case'' situations, which rarely occur in processor's real-world usage scenarios. To reclaim the wasted timing margin, we study \textit{active timing margin}, a runtime technique that dynamically adjusts pipeline timing margin to match real-time load conditions. 

%We propose synergistic management schemes that involve circuit, architecture, and system software techniques to safely accommodate various load conditions and to maximize active timing margin's potential benefits. At circuit level, a set of sensors are needed to provide cycle-level information of pipeline timing margin. At architecture level, hardware and software techniques are implemented to address different load conditions. In system software, an intelligent application mapper help maximize the gains of active timing margin. 

