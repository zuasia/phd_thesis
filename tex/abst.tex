%!TEX root=../paper.tex

% The abstract is required. Note the use of ``utabstract'' instead of
% ``abstract''! This was necessary to fix a page numbering problem.
% The abstract heading is generated automatically.
% Do NOT use \begin{abstract} ... \end{abstract}.
%
\utabstract
\indent

Improving efficiency is critical for today's microprocessors. From smartphones to datacenters, high power/performance efficiency always entails better systems, measured by longer battery time or lower total cost of ownership. In my dissertation, I propose improving microprocessor power/performance efficiency by optimizing the pipeline timing margin. In particular, I focus on investigating the fundamentals and efficacy of \textit{Active Timing Margin}, a technique that automatically adjusts pipeline timing margin according to the processor's real-time load environment. The key insight is that in order to maximize \textit{Active Timing Margin}'s efficiency enhancement benefit, synergistic management from processor architecture design and system software scheduling are needed. To that end, this dissertation's study on \textit{Active Timing Margin} covers the major effects that determines pipeline timing margin, including temperature, voltage, and process variation. This dissertation is likely to have long-term impact because the work is conducted based on solid measurement on state-of-the-art processors, and the research target, \textit{Active Timing Margin}, can benefit virtually all processor architectures, and it already has wide applicability in the latest microprocessors shipped by the time this dissertation is written.


%The problem we address is that the timing margin in conventional microprocessors is a static value determined during chip design stage, and is significantly over-provisioned to protect against hypothetical ``worst-case'' situations, which rarely occur in processor's real-world usage scenarios. To reclaim the wasted timing margin, we study \textit{active timing margin}, a runtime technique that dynamically adjusts pipeline timing margin to match real-time load conditions. 

%We propose synergistic management schemes that involve circuit, architecture, and system software techniques to safely accommodate various load conditions and to maximize active timing margin's potential benefits. At circuit level, a set of sensors are needed to provide cycle-level information of pipeline timing margin. At architecture level, hardware and software techniques are implemented to address different load conditions. In system software, an intelligent application mapper help maximize the gains of active timing margin. 

